 
Despu\'{e}s de haber realizado el procedimiento descrito en los anteriores cap\'{i}tulos para generar un mecanismo (modelo) capaz de detectar anomal\'{i}as de conducci\'{o}n, se puede decir que se ha cumplido a cabalidad con los objetivos propuestos en el proyecto. A continuaci\'{o}n se presentar\'{a} las conclusiones a las que se llego as\'{i} como tambi\'{e}n aquellas nuevas ideas e inquietudes que surgieron durante el proceso de desarrollo, las cuales podr\'{i}an mejorar los resultados obtenidos por el presente trabajo de grado.


\section{Conclusiones}

La presente investigaci\'{o}n se ha dedicado al 


\section{Limitaciones y recomendaciones}

Una vez concluido el trabajo de grado, se considera interesante investigar sobre aspectos para la detecci\'{o}n de anomal\'{i}as y se propone:

\begin{itemize}
\item Agregar la velocidad del veh\'{i}culo como un nuevo par\'{a}metro del conjunto de datos, debido a que esto podr\'{i}a brindar un mejor entendimiento del comportamiento normal, as\'{i} como tambi\'{e}n de las anomal\'{i}as.
\item Utilizar tensorflow 2.0 en lugar del framework keras 
\end{itemize}

usar datos que son invriantes al movimiento del celular



realizar mas preubas con otras anomalias

