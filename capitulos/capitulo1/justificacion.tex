
\section{Justificación}

Los accidentes de tránsito cobran un número inaceptable de víctimas cada a\~{n}o, especialmente en las regiones m\'{a}s pobres del mundo. Esto se debe a diversos aspectos, pero el principal recae en el bajo nivel de conciencia ciudadana que existe, lo que conlleva a que muchas personas manejen bajo los efectos del alcohol, con exceso de velocidad, manipulando sus dispositivos m\'{o}viles, entre otros. Por ello este trabajo busca establecer patrones de comportamientos de conducci\'{o}n mediante el uso de un dispositivo m\'{o}vil y t\'{e}cnicas de Aprendizaje Autom\'{a}tico, de manera que se logre realizar una detecci\'{o}n de anomal\'{i}as de manejo oportuna.

\subsection{Justificaci\'{o}n pr\'{a}ctica}

Detectar anomal\'{i}as de conducci\'{o}n permite generar una alerta oportuna al conductor para que \'{e}ste logre corregir sus conductas de manejo de forma r\'{a}pida. De esta manera se podr\'{a} evitar accidentes de tr\'{a}nsito o minimizar los efectos del mismo, permitiendo as\'{i} reducir la cantidad de dan\~{o}s, tanto materiales como personales.

\subsection{Justificaci\'{o}n metodol\'{o}gica}

El estudio realizado en el desarrollo del presente trabajo de grado permite resaltar la eficiencia de las t\'{e}cnicas de Aprendizaje Autom\'{a}tico en la detecci\'{o}n de anomal\'{i}as.

\section{L\'{i}mites y alcances}

Debido a que la realizaci\'{o}n de pruebas de campo para \'{e}sta investigaci\'{o}n es bastante peligrosa, se limit\'{o} los ejemplos de conducci\'{o}n an\'{o}mala a: .......

\vspace{5mm} %5mm vertical space

Siendo as\'{i} los experimentos y pruebas se realizaron s\'{o}lo sobre un peque\~{n}o conjunto de ejemplos an\'{o}malos, por lo tanto no se espera que \'{e}ste funcione de manera correcta sobre aquellos ejemplos que no fueron considerados.

\section{M\'{e}todo de investigaci\'{o}n}

La presente investigaci\'{o}n se realiz\'{o} con un enfoque experimental, teniendo como hip\'{o}tesis la siguiente:

