\chapter{\uppercase{CONCLUSIONS AND FUTURE WORK}}
\label{Capitulo 7}

After having carried out procedure described in previous chapters, in order to verify the hypothesis established in this investigation, a mechanism (model) capable of detecting conduction anomalies was generated. In this way it can be said that the objectives proposed in this research have been fully met. Next, conclusions reached will be presented, as well as those new ideas and concerns that arose during development process, which could improve results obtained by this work.
%Despu\'{e}s de haber realizado el procedimiento descrito en los anteriores cap\'{i}tulos, con el objetivo de comprobar la hip\'{o}tesis establecida en la presente investigaci\'{o}n, se gener\'{o} un mecanismo (modelo) capaz de detectar anomal\'{i}as de conducci\'{o}n. De esta forma se puede decir que se ha cumplido a cabalidad con los objetivos propuestos en esta investigaci\'{o}n. A continuaci\'{o}n se presentar\'{a} las conclusiones a las que se lleg\'{o}, as\'{i} como tambi\'{e}n aquellas nuevas ideas e inquietudes que surgieron durante el proceso de desarrollo, las cuales podr\'{i}an mejorar los resultados obtenidos por el presente trabajo.

\section{Conclusions}

The main objective of this research work was to develop a mechanism capable of detecting conduction anomalies, such that, it is provided with a solution to alert in a timely manner the finding of anomalous patterns in the conduction of agents, whether human or autonomous, independent each model according to experience and environment through which each agent traveled.
%El objetivo fundamental de este trabajo de investigaci\'{o}n fue desarrollar un mecanismo capaz de detectar anomal\'{i}as de conducci\'{o}n, tal que, se aporte con una soluci\'{o}n para alertar de forma oportuna el hallazgo de patrones an\'{o}malos en la conducci\'{o}n de agentes, ya sean humanos o aut\'{o}nomos, independizando cada modelo seg\'{u}n la experiencia y el ambiente por el que recorre cada agente.

\vspace{5mm} %5mm vertical space

Thus, the main contribution of this study consists in the implementation of a mechanism capable of identifying anomalies from normal conduction data of each agent, without human intervention, that is, detector model does not require a human to intervene to generate it. However, this can be optimized by adjusting the \textit{Contamination} hyperparameter in order to define how sensitive to anomalies this detector will be. On the other hand, it can be said that the detection mechanism of this research work, in addition to being novel, is one of few works that were carried out with a \textit{''semi-supervised''} approach, since most of work carried out at date was done using a supervised approach.
%As\'{i} pues, el principal aporte de este estudio consiste en la implementaci\'{o}n de un mecanismo capaz de identificar anomal\'{i}as a partir de los datos de conducci\'{o}n normal de cada agente, sin intervenci\'{o}n humana, es decir, el modelo detector no requiere que un humano intervenga para generarlo, sin embargo, este puede ser optimizado por medio del ajuste del hiperpar\'{a}metro \textit{Contaminaci\'{o}n} con el fin de definir cu\'{a}n sensible a las anomal\'{i}as ser\'{a} dicho detector. Por otra parte, se puede decir que el mecanismo de detecci\'{o}n de este trabajo de investigaci\'{o}n, adem\'{a}s de ser novedoso, es uno de los pocos trabajos que se realizaron con un enfoque \textit{''semi-supervisado''}, ya que la mayor\'{i}a de los trabajos realizados a la fecha fueron realizados mediante un enfoque supervisado.

\vspace{5mm} %5mm vertical space

Conclusions derived from this research work were made based on different experiments carried out, these conclusions are set out below.
%Las conclusiones que se derivan de este trabajo de investigaci\'{o}n se hicieron en base a los diferentes experimentos realizados, dichas conclusiones se exponen a continuaci\'{o}n.

\begin{itemize}
\item It is verified, from results' analysis of this study, the capacity with which the inertial sensors of a mobile device have to represent correctly a car's movement and thus be able to feed, with a previous pre-processing, an anomaly detection mechanism.
%\item Se comprueba, a partir del an\'{a}lisis de resultados de este estudio, la capacidad con la que cuentan los sensores inerciales de un dispositivo m\'{o}vil para representar correctamente el movimiento de un autom\'{o}vil y de esa manera ser capaz de alimentar, con un previo pre-procesamiento, un mecanismo de detecci\'{o}n de anomal\'{i}as.
\item In this work, different neural network architectures were compared to generate the normal behavior's model, where the simplest network achieved the best results both in precision and in the time used during prediction process; thus demonstrating that more complex networks do not always interpret data sets better.
%\item En este trabajo se compararon diferentes arquitecturas de redes neuronales para generar el modelo del comportamiento normal, donde la red m\'{a}s simple logr\'{o} los mejores resultados tanto en precisi\'{o}n como en el tiempo empleado durante el proceso de predicci\'{o}n; demostrando as\'{i}, que no siempre las redes m\'{a}s complejas interpretan mejor los conjuntos de datos.
\item On the other hand, different techniques were compared to define an anomaly detection method appropriate to the present investigation's context, where due to the simplicity of its training and its robust result, the option of isolation forest technique was chosen, with a value of 0.0075 for \textit{Contamination} hyperparameter.
%\item Por otra parte, se compar\'{o} diferentes t\'{e}cnicas para definir un m\'{e}todo de detecci\'{o}n de anomal\'{i}as adecuado al contexto de la presente investigaci\'{o}n, donde por la simplicidad de su entrenamiento y por su robusto resultado se opt\'{o} por la elecci\'{o}n de la t\'{e}cnica de bosques de aislamiento, con un valor de 0.0075 para el hiperpar\'{a}metro \textit{Contaminaci\'{o}n}.
\item By integrating normal behavior model and anomaly detection method, which were only trained with the ''normal'' data set, creation of a mechanism capable of identifying outliers of conduction for each agent is achieved.
%\item Integrando el modelo del comportamiento normal y el m\'{e}todo de detecci\'{o}n de anomal\'{i}as, los cuales s\'{o}lo fueron entrenados con el conjunto de datos ''normal'', se logra la creaci\'{o}n de un mecanismo capaz de identificar valores at\'{i}picos de la conducci\'{o}n de cada agente.
\item Finally, the capacity of detection mechanism was evaluated, through evaluation set which presents anomalous samples, resulting in the correct detection of 67.68\% of samples which present anomalies in evaluation set, as well as presenting a rate of only 0.80\% of normal samples detected as anomalies. Being a breakthrough in anomaly detection's field with a semi-supervised approach.
%\item Finalmente se evalu\'{o} la capacidad del mecanismo de detecci\'{o}n, mediante el conjunto de evaluaci\'{o}n el cu\'{a}l presenta muestras an\'{o}malas, dando como resultado la correcta detecci\'{o}n del 67.68\% de las muestras, que presentan anomal\'{i}as en el conjunto de evaluaci\'{o}n, as\'{i} como tambi\'{e}n presenta una tasa de tan s\'{o}lo 0.80\% de muestras normales detectadas como anomal\'{i}as. Siendo un gran avance en el \'{a}mbito de la detecci\'{o}n de anomal\'{i}as con un enfoque semi-supervisado.%, ya que a pesar de aparentar una precisi\'{o}n muy baja detecta por lo menos algunas de las muestras (dato por segundo) de cada anomal\'{i}a presente en el conjunto de evaluaci\'{o}n, lo cual podr\'{i}a interpretarse que este mecanismo detecta la totalidad de las anomal\'{i}as del conjunto con el que se evalu\'{o}.
\end{itemize}

This research work, before presenting a final solution, lays the foundations for development of detection systems for agents' conduction anomalies, through  use of Artificial Intelligence techniques, highlighting capacity and scope of this study, since it not only focuses on driving human agents, but is just as capable of being applied in an autonomous driving approach.
%Este trabajo de investigaci\'{o}n antes que presentar una soluci\'{o}n final sienta las bases para el desarrollo de sistemas de detecci\'{o}n de anomal\'{i}as de la conducci\'{o}n de los agentes, mediante el uso de t\'{e}cnicas de Inteligencia Artificial, resaltando la capacidad y alcance que conlleva este estudio, ya que no s\'{o}lo se enfoca en la conducci\'{o}n de agentes humanos, sino que es igual de capaz de ser aplicado en un enfoque de conducci\'{o}n aut\'{o}nomo.

\section{Future works}

Once research work is concluded, it is considered interesting to investigate different aspects of anomaly detection and it is proposed:
%Una vez concluido el trabajo de investigaci\'{o}n, se considera interesante investigar sobre diferentes aspectos de la detecci\'{o}n de anomal\'{i}as y se propone:

\begin{itemize}
\item Add vehicle speed as a new parameter in dataset, as this could provide a better understanding of normal driving behavior as well as anomalies.
%\item Agregar la velocidad del veh\'{i}culo como un nuevo par\'{a}metro del conjunto de datos, debido a que esto podr\'{i}a brindar un mejor entendimiento del comportamiento normal de conducci\'{o}n, as\'{i} como tambi\'{e}n de las anomal\'{i}as.
\item Instead of working with raw data, use difference between a data captured at time $t$ and a data captured at $t-1$ ($dif_{t} = dato_{t}-dato_{t-1}$), applying this data preprocessing could maximize detection of those anomalies that present high differences between consecutive data.
%\item En lugar de trabajar con los datos en crudo, usar la diferencia entre un dato capturado en el tiempo $t$ y un dato capturado en $t-1$ ($dif_{t} = dato_{t}-dato_{t-1}$), la aplicaci\'{o}n de \'{e}ste pre-procesamiento de datos podr\'{i}a maximizar la detecci\'{o}n de aquellas anomal\'{i}as que presentan elevadas diferencias entre los datos consecutivos.
\item Validate model with new types of anomalies such as: drift, crash, high-speed U-turn, among others. This was due to the fact that study was limited to recognition of only three types of anomalies due to the difficulty and danger associated with their capture.
%\item Validar el modelo con nuevos tipos de anomal\'{i}as como por ejemplo: derrapes, choques, giros en U a alta velocidad, entre otros. Esto debido a que el estudio se limit\'{o} al reconocimiento de s\'{o}lo tres tipos de anomal\'{i}as por la dificultad y peligro que conlleva su captura.
%\item Probar si el modelo propuesto incrementa su precisi\'{o}n en caso de aumentar la cantidad del conjunto de datos.
\item Extend the model so that it is capable of determining not only an anomaly but also the type to which said anomaly belongs.
%\item Extender el modelo para que sea capaz de determinar no s\'{o}lo una anomal\'{i}a sino tambi\'{e}n el tipo al que dicha anomal\'{i}a pertenece.
\item Migrate the normal behavior model from keras to Tensorflow 2.0.
%\item Migrar el modelo del comportamiento normal de keras a Tensorflow 2.0.
\item Implement an information system to monitor anomalies using the detection mechanism proposed in this work.
%\item Implementar un sistema de informaci\'{o}n para monitorear las anomal\'{i}as mediante el mecanismo de detecci\'{o}n propuesto en el presente trabajo.
\end{itemize}

 

