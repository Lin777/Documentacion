%\chapter{Bibliografia}
%\bibliographystile{plain}

\begin{thebibliography}{9}
%%%%%%%%%%%%%%%%%%%%%%%%%%%%%%%%%%%%%%%
%Desde aqui empieza la verdadera bibliografia
%%%%%%%%%%%%%%%%%%%%%%%%%%%%%%%%%%%%%%
\bibitem{20}
DANG-NHAC, L., DUC-NHAN, N., THI-HAU, N., y HA-NAM, N. Vehicle mode and driving activity detection based on analyzing sensor data of smartphones, \textit{Sensors} v. 18, 4 (2018), 1036.

\bibitem{21}
FERREIRA, J., CARVALHO, E., FERREIRA, B., DE SOUZA, C., SUHARA, Y., PENTLAND, A., y PESSIN, G. Driver behavior profiling: An investigation with different smartphone sensors and machine learning, \textit{PLoS One} v. 12, 4 (2017), e0174959.

\bibitem{22}
WHO-LEE, K., SIK-YOON, H., MIN-SONG, J., y RYOUNG-PARK, K. Convolutional Neural Network-Based Classification of Driver’s Emotion during Aggressive and Smooth Driving Using Multi-Modal Camera Sensors, \textit{Sensors} v. 18, 4 (2018), 957.

\bibitem{23}
JOHNSON, D., y TRIVEDI, M. Driving Style Recognition Using a Smartphone as a Sensor Platform. En: Intelligent Transportation Systems (ITSC), \textit{14th International IEEE Conference}, 2011. pp. 1609–1615.

\bibitem{24}
KRIDALUKMANA, R., YAN-LU, H., y NADERPOUR, M. An Object Oriented Bayesian Network Approach for Unsafe Driving Maneuvers Prevention System, \textit{12th International IEEE Conference}, 2017.

\bibitem{25}
BHOYAR, V., LATA, P., KATKAR, J., PATIL, A., y JAVALE, D. Symbian Based Rash Driving Detection System. Int. J. Emerg. Trends Technol. Comput. Sci. 2013; 2:124–126.

\bibitem{26}
CHEN, Z., YU, J., ZHU, Y., CHEN, Y., y LI, M. D3: Abnormal Driving Behaviors Detection and Identification Using Smartphone Sensors, \textit{Proceedings of the 12th Annual IEEE International Conference on Sensing, Communication, and Networking}; Seattle, WA, USA. 22–25 de Junio 2015; pp. 524–532.

\bibitem{27}
EREN, H., MAKINIST, S., AKIN, E., y YILMAZ, A. Estimating Driving Behavior by a Smartphone, \textit{Proceedings of the Intelligent Vehicles Symposium}; Alcalá de Henares, Espa\~{n}a. 3–7 de Junio 2012; pp. 234–239.

\bibitem{28}
BOONMEE, S., y TANGAMCHIT, P. Portable Reckless Driving Detection System; Proceedings of the 6th IEEE International Conference on Electrical Engineering/Electronics, Computer, Telecommunications and Information Technology; Pattaya, Chonburi, Tailandia. 6–9 de Mayo 2009; pp. 412–415.

\bibitem{29}
KOH, D.-W., y KANG, H.-B. Smartphone-Based Modeling and Detection of Aggressiveness Reactions in Senior Drivers; \textit{Proceedings of the IEEE Intelligent Vehicles Symposium}; Seoul, Korea. 28 de Junio – 1 de Julio 2015; pp. 12–17.

\bibitem{30}
ZALDIVAR, J., CALAFATE, C.T., CANO, J.C., y MANZONI, P. Providing accident detection in vehicular networks through OBD-II devices and Android-based smartphones; \textit{Proceedings of the 2011 IEEE 36th Conference on Local Computer Networks}; Bonn, Alemania. 4–7 de Octubre 2011; pp. 813–819.

\bibitem{31}
ARAUJO, R., IGREJA, A., DE CASTRO R., y ARAUJO, R. Driving coach: A smartphone application to evaluate driving efficient patterns; \textit{Proceedings of the 2012 IEEE Intelligent Vehicles Symposium (IV)}; Alcala de Henares, Espa\~{n}a. 3–7 de Junio 2012; pp. 1005–1010.

\bibitem{32}
SHAI, S., y SHAI, B. Understanding Machine Learning: From Theory to Algorithms (2014).

\bibitem{33}
XUE, Z., SHANG, Y., y FENG, A. Semi-supervised outlier detection based on fuzzy rough C-means clustering. \textit{Mathematics and Computers in Simulation}, vol. 80, no. 9, pp. 1911–1921.

\bibitem{34}
RUMELHART, D. E., HINTON, G. E., y WILLIAMS, R. J. (1986). Learning representations by backpropagating errors. \textit{Nature}, 323(6088), 533–536.

\bibitem{35}
ELMAN, J. Finding structure in time. Cognitive Science (1990), 14(2), 179–211.

\bibitem{36}
WERBOS, P. J. Generalization of backpropagation with application to a recurrent gas market model. \textit{Neural Networks} (1988), 1(4), pp. 339–356.

%https://books.google.com.bo/books?hl=en&lr=&id=hhdVr9F-JfAC&oi=fnd&pg=PR17&dq=preparation+of+data+machine+learning+pdf&ots=6gfXbLOB9r&sig=MwromFyjhNbymZUVbc_K0krLJm8&redir_esc=y#v=onepage&q=preparation&f=false
\bibitem{37}
PYLE, D. Data preparation for data mining (1999), pp. 90.

%http://docsdrive.com/pdfs/medwelljournals/jeasci/2017/4102-4107.pdf
\bibitem{38}
SUAD, A., WESAM, S. Review of Data Preprocessing Techniques in Data Mining. \textit{Journal of Engineering and Applied Sciences} 12(16): 4102-4107. 2017.

%https://drive.google.com/drive/folders/1qlqVkpZ__Wb2cIzkqZyv2scJP0ijCNE1
\bibitem{39}
BISHOP, M. C. Pattern recognition and Machine Learning (2006), pp. 561.

\bibitem{40} BELLMAN, R. E. Dynamic programming (2003). Courier Dover Publications. \textit{ISBN} 978-0-486-42809-3.

\bibitem{41} MOINDROT, O. y GENTHIAL, G. Splitting into train, dev and test sets. 24 de Enero en 2018. Disponible en: \texttt{https://cs230-stanford.github.io/train-dev-test-split.html}.

% https://cocosci.princeton.edu/tom/papers/discgencat.pdf
\bibitem{42} HSU, A. y GRIFFITHS, T. Effects of generative and discriminative learning on use of category variability.

\bibitem{43} DAUPHIN, Y. N., FAN, A., AULI, M., y GRANGIER, D. Language Modeling with Gated Convolutional Networks, \textit{arXiv}, 2017.

\bibitem{44} MASS, A., HANNUN, A., y NG, A., Rectifier Nonlinearities Improve Neural Network Acoustic Models. \textit{International Conference on Machine Learning (icml)}, 2013.

\bibitem{45} ZEILER, M. D., RANZATO, M.,MONGA, R.,MAO, M.,YANG, K., LE, Q. V., y HINTON, G. E. On rectified linear units for speech processing. \textit{International Conference on Acoustics, Speech and Signal Processing. IEEE}, 2013, pp. 3517–3521, IEEE.

% https://doi.org/10.1038/nature14539
\bibitem{46} LECUN, Y., BENGIO, Y., y HINTON, G. Deep learning. \textit{Nature}, vol. 521, no. 7553, pp. 436–444, 2015.

% bibliografia de las funciones de activacion => https://arxiv.org/pdf/1811.03378.pdf

% http://proceedings.mlr.press/v95/guo18a/guo18a.pdf
\bibitem{47} GUO, Y., LIAO, W., WANG, Q., YU, L., JI, T. y LI, P. Multidimensional Time Series Anomaly Detection: A GRU-based Gaussian Mixture Variational Autoencoder Approach. \textit{Proceedings of Machine Learning Research} 95:97-112, 2018.

%http://yann.lecun.com/exdb/publis/pdf/lecun-89c.pdf
\bibitem{48} LECUN, Y., JACKEL, L., BOSER, B., DENKER, J., GRAF, H., GUYON, I., HENDERSON, D., HOWARD, R., y HUBBARD, W. Handwritten digit recognition : Applications of neural networks chips and automatic learning. \textit{Proceedings of the IEEE}, 86(11):2278–2324, 1998.

% http://link.springer.com/article/10.1007/s11263-015-0816-y#
\bibitem{49} RUSSAKOVSKY, O. et al. ImageNet Large Scale Visual Recognition Challenge. \textit{International Journal of Computer Vision}. 2014. 

% https://ieeexplore.ieee.org/document/279181/authors#authors
\bibitem{50} BENGIO, Y., SIMARD, P., FRASCONI, P. Learning long-term dependencies with gradient descent is difficult. \textit{IEEE Trans. Neural Netw}. 1994, 5, 157–166.

\bibitem{51} PASCANU, R., MIKOLOV, T., BENGIO, Y. On the difficulty of training recurrent neural networks. \textit{In Proceedings of the International Conference on Machine Learning}, Atlanta, GA, USA, 16–21 Junio de 2013; pp. 1310–1318.

\bibitem{52} HOCHREITER, S., SCHMIDHUBER, J. Long short-term memory. \textit{Neural Comput}. 1997, 9, 1735–1780.

\bibitem{53} OLAH, C. Understanding LSTM Networks. Disponible en la p\'{a}gina web: \texttt{http://colah.github.io/posts/2015-08-Understanding-LSTMs/} (\'{U}ltimo acceso en 11 de octubre de 2019).

\bibitem{54} YAN, S. Understanding LSTM and Its Diagrams. Disponible en la p\'{a}gina web: \texttt{https://medium.com/mlreview/understanding-lstm-and-its-diagrams-37e2f46f1714} (\'{U}ltimo acceso en 11 de octubre de 2019).

\bibitem{55} CHO, K., MERRI\"{E}NBOER, B. V., GULCEHRE, C., BAHDANAU, D., BOUGARES, F., SCHWENK, H., y BENGIO, Y. Learning phrase representations using rnn encoder-decoder for statistical machine translation. \textit{arXiv} preprint arXiv:1406.1078, 2014a.

\bibitem{56} ZHANG, A., LIPTON, Z., LI, M., y SMOLA, A. Dive into Deep Learning, 25 de Septiembre de 2019. Disponible en la p\'{a}gina web: \texttt{https://en.d2l.ai/d2l-en.pdf} (\'{U}ltimo acceso en 11 de octubre de 2019).

\bibitem{57} HAWKINS, S., HE, H., WILLIAMS, G. y BAXTER, R. Outlier Detection Using Replicator Neural Networks. \textit{International Conference on Data Warehousing and Knowledge Discovery}. pp. 170–180. No. September, Springer Berlin Heidelberg (2002).

\bibitem{58}  WILLIAMS, G. y BAXTER, R. A comparative study of RNN for outlier detection in data mining. \textit{IEEE International Conference on Data Mining} (December 2002), 1–16 (2002).

\bibitem{59} SCH\"{O}LKOPF, B. y SMOLA, A. J. Support vector machines, regularization, optimization, and beyond. \textit{MIT Press}, pp. 656:657, 2002.

\bibitem{60} WOLPHER, M. Anomaly Detection in Unstructured Time Series Data using an LSTM Autoencoder. Disponible en la p\'{a}gina web: \texttt{http://www.diva-portal.org/smash/get/diva2:1225367/FULLTEXT01.pdf} (\'{U}ltimo acceso en 11 de octubre de 2019).

\bibitem{61} SMITS, P., DELLPIANE, S., y SCHOWENGERDT, R. Quality assessment of image classification algorithms for land-cover mapping: a review and a proposal for a cost-based approach. \textit{International journal of remote sensing 20}, 8 (1999), pp.1461–1486.


\bibitem{62} \"{O}ZLER, H. Accuracy Trap! Pay Attention to Recall, Precision, F-Score, AUC. Disponible en la p\'{a}gina web: \texttt{https://medium.com/datadriveninvestor/accuracy-trap-pay-attention-to-recall-precision-f-score-auc-d02f28d3299c} (\'{U}ltimo acceso en 16 de octubre de 2019).

\end{thebibliography} 
