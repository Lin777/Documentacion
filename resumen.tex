\bigskip
El presente trabajo describe el desarrollo de un mecanismo de detecci\'{o}n de anomal\'{i}as de conducci\'{o}n, el cual es implementado usando un dispositivo m\'{o}vil y t\'{e}cnicas de Aprendizaje Autom\'{a}tico. 

El objetivo es crear una herramienta capaz de encontrar comportamientos an\'{o}malos en la conducci\'{o}n de un agente humano o aut\'{o}nomo, teniendo un previo conocimiento de las conductas normales de conducci\'{o}n del mismo. Asimismo se presenta antecedentes de trabajos e investigaciones de la detecci\'{o}n de anomal\'{i}as de conducci\'{o}n de todo el mundo, se analiza los par\'{a}metros de conducci\'{o}n obtenidos por el dispositivo m\'{o}vil, y se presenta la propuesta para identificar anomal\'{i}as mediante el uso de Redes Neuronales y Bosques de Aislamiento, un m\'{e}todo de Aprendizaje Autom\'{a}tico que es comunmente utilizado para la detecci\'{o}n de anomal\'{i}as.

El trabajo cuenta con dos partes principales: un modelo ajustado al comportamiento normal de conducci\'{o}n de un agente, y un m\'{e}todo de detecci\'{o}n de anomal\'{i}as, los cuales fueron entrenados iterativamente con 30000 muestras, las cuales corresponden s\'{o}lo al comportamiento normal de conducci\'{o}n.

La precisi\'{o}n de detecci\'{o}n del mecanismo completo propuesto en este documento, es de 67.68\% que fue evaluado con 44040 datos, de los cuales 164 corresponden a muestras an\'{o}malas, siendo as\'{i} una de las contribuciones m\'{a}s sobresalientes para la detecci\'{o}n de anomal\'{i}as de conducci\'{o}n semi-supervisada.


%El presente trabajo plantea qué, con la captura de parámetros de manejo de un conductor mediante el uso de un dispositivo móvil, es posible encontrar patrones de conducción que describan diferentes comportamientos de manejo; y de esta manera al encontrar un patrón anómalo solicitar al conductor que compruebe que es hábil para seguir conduciendo y en caso de que éste no logre hacerlo, en un determinado tiempo, notificar de esta irregularidad a los contactos de emergencia del conductor y así prevenir y/o reducir conductas de manejo riesgosas.