\bigskip

This paper describes the development of a mechanism to detect driving anomalies, which is implemented using a mobile device and Machine Learning techniques.

%El presente trabajo describe el desarrollo de un mecanismo de detecci\'{o}n de anomal\'{i}as de conducci\'{o}n, el cual es implementado usando un dispositivo m\'{o}vil y t\'{e}cnicas de Aprendizaje Autom\'{a}tico. 

The objective is create a tool capable to find anomalous behaviors in the driving of human or autonomous agent, having a previous knowledge of normal driving conducts of them. Likewise it presents a background of driving anomalies detection's researches around the world, the driving parameters obtained by the mobile device are analized, and a proposal is presented to identify anomalies through the use of Neural Networks and Isolation Forests, a method of Machine Learning that is commonly used for anomalies' detection. 

%El objetivo es crear una herramienta capaz de encontrar comportamientos an\'{o}malos en la conducci\'{o}n de un agente humano o aut\'{o}nomo, teniendo un previo conocimiento de las conductas normales de conducci\'{o}n del mismo. Asimismo se presenta antecedentes de trabajos e investigaciones de la detecci\'{o}n de anomal\'{i}as de conducci\'{o}n de todo el mundo, se analiza los par\'{a}metros de conducci\'{o}n obtenidos por el dispositivo m\'{o}vil, y se presenta la propuesta para identificar anomal\'{i}as mediante el uso de Redes Neuronales y Bosques de Aislamiento, un m\'{e}todo de Aprendizaje Autom\'{a}tico que es com\'{u}nmente utilizado para la detecci\'{o}n de anomal\'{i}as.

The work has two main parts: a model adjusted to the normal driving behavior of an agent, and a method of detecting anomalies, which were iteratively trained with 30,000 samples, which correspond only to normal driving behavior.

%El trabajo cuenta con dos partes principales: un modelo ajustado al comportamiento normal de conducci\'{o}n de un agente, y un m\'{e}todo de detecci\'{o}n de anomal\'{i}as, los cuales fueron entrenados iterativamente con 30000 muestras, las cuales corresponden s\'{o}lo al comportamiento normal de conducci\'{o}n.

The detection accuracy of the complete mechanism proposed in this document is 67.68\%, which was evaluated with 44040 samples, of which 164 correspond to anomalous samples, thus being one of the most outstanding contributions for the detection of driving anomalies with a semi supervised approach.

%La precisi\'{o}n de detecci\'{o}n del mecanismo completo propuesto en este documento, es de 67.68\% que fue evaluado con 44040 datos, de los cuales 164 corresponden a muestras an\'{o}malas, siendo as\'{i} una de las contribuciones m\'{a}s sobresalientes para la detecci\'{o}n de anomal\'{i}as de conducci\'{o}n semi-supervisada.
