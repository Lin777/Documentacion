El presente trabajo describe el desarrollo de un mecanismo de detecci\'{o}n de anomal\'{i}as de manejo, el cual es implementado usando un dispositivo m\'{o}vil y t\'{e}cnicas de Inteligencia Artificial (IA). 

El objetivo es crear una herramienta capaz de encontrar comportamientos an\'{o}malos en el manejo de un agente humano o aut\'{o}nomo, teniendo un previo conocimiento de las conductas normales de manejo del mismo. Asimismo se presenta antecedentes de trabajos e investigaciones de la detecci\'{o}n de anomal\'{i}as de manejo de todo el mundo, se analiza los par\'{a}metros de manejo de conducci\'{o}n obtenidos por el dispositivo m\'{o}vil, y se presenta la propuesta para identificar anomal\'{i}as mediante Redes Neuronales Recurrentes, un m\'{e}todo de Aprendizaje Autom\'{a}tico que es comunmente utilizado para series de tiempo.

El trabajo cuenta con dos partes principales: un modelo ajustado al comportamiento normal de manejo de un agente, y un m\'{e}todo de detecci\'{o}n de anomal\'{i}as.....

La precisi\'{o}n de detecci\'{o}n del m\'{e}todo proupuesto en este proyecto es de .... que fue evaluado con .... 





%El presente trabajo plantea qué, con la captura de parámetros de manejo de un conductor mediante el uso de un dispositivo móvil, es posible encontrar patrones de conducción que describan diferentes comportamientos de manejo; y de esta manera al encontrar un patrón anómalo solicitar al conductor que compruebe que es hábil para seguir conduciendo y en caso de que éste no logre hacerlo, en un determinado tiempo, notificar de esta irregularidad a los contactos de emergencia del conductor y así prevenir y/o reducir conductas de manejo riesgosas.